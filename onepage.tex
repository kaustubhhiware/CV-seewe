%%%%%%%%%%%%%%%%%%%%%%%%%%%%%%%%%%%%%%% 
% Deedy CV/Resume
% XeLaTeX Template
% Version 1.0 (5/5/2014)
% 
% This template has been downloaded from:
% http://www.LaTeXTemplates.com
% 
% Original author:
% Debarghya Das (http://www.debarghyadas.com)
% With extensive modifications by:
% Vel (vel@latextemplates.com)
% 
% License:
% CC BY-NC-SA 3.0 (http://creativecommons.org/licenses/by-nc-sa/3.0/)
% 
% Important notes:
% This template needs to be compiled with XeLaTeX.
% 
%%%%%%%%%%%%%%%%%%%%%%%%%%%%%%%%%%%%%% 

\documentclass[a4paper]{deedy-resume} % Use US Letter paper, change to a4paper for A4 

\newcommand{\onepage}[1]{
  \ifdefined\onep
  #1
  \fi
}

\begin{document}
\namesection{Kaustubh}{Hiware}{
  Computer Science and Engineering, IIT Kharagpur \\
  --\\
  \href{mailto:hiwarekaustubh@iitkgp.ac.in}{ hiwarekaustubh@iitkgp.ac.in}  |  (+91) 8768829567\\
  Indian Institute of Technology Kharagpur - 721302}
\begin{minipage}[t]{0.33\textwidth} % The left column takes up 33% of the text width of the page

  \section{Education} 

  \descriptbig{IIT Kharagpur}

  \descript{B.Tech+M.Tech in Computer Science \\ and Engineering}
  \location{2014-Present | CGPA: \textit{8.05/10}}

  \sectionspace
  \sectionspace
  
  \descriptbig{Shivaji Science College, \\ Nagpur(State Board)}
  \descript{Senior Secondary Grade 12}
  \location{Overall: \textbf{95.54\%}}

  \sectionspace
  \sectionspace
  
  \descriptbig{Sandipani School, \\ Nagpur (CBSE)}
  \descript{Secondary Grade 10}
  \location{Overall: \textbf{96.8\%}}    
    
  \sectionspace

  \section{Links} 

  Github:// \href{https://github.com/kaustubhhiware}{\bf kaustubhhiware} \\
  LinkedIn:// \href{https://www.linkedin.com/in/kaustubhhiware}{\bf kaustubhhiware} \\
  Website:// \href{http://kaustubhhiware.github.io/}{\bf kaustubhhiware} \\
  
  \sectionspace

  \section{CS Coursework}
  Algorithms 1, Algorithms 2 *\\
  Software Engineering \\
  Computer Organization *\\
  Programming \& Data Structures\\
  Compilers *, Automata Theory\\
  Discrete Structures\\
  Matrix Algebra *, Data Analytics *\\
  Probability and Statistics\\
  Switching Circuit and Logic \\

  {\footnotesize \textit{\textbf{(* denotes ongoing courses) }}} \\

  \sectionspace

  \section{Skills}

  \runsubsection{}
  \descriptbig{Programming Languages}
  C/C++ \textbullet{} Java \textbullet{} Python \\
  \textbullet{} x86 Assembly

  \sectionspace
  \sectionspace

  \descriptbig{Operating Systems}
  Ubuntu \textbullet{} Microsoft Windows

  \sectionspace
  \sectionspace

  \descriptbig{Utilities}
  Git \textbullet{} Linux Shell  \\
  \textbullet{} OpenCV

  \sectionspace
  \sectionspace

  \section{Interests}
  Data Analytics \\
  Machine Learning \\
  Algorithms \\
  NLP \\
  Open source \\
\end{minipage}
\hfill
\begin{minipage}[t]{0.66\textwidth}

  \section{Awards and Achievements}

\fontspec[Path = fonts/raleway/]{Raleway-Light}\fontsize{11pt}{11pt}\selectfont {

  \begin{tabular}{rll}
    2014	 & \bold{All India Rank 1874}, JEE Advanced, among 150,000 candidates\\
    2014	 & \bold{All India Rank 277}, JEE Mains(I), among 1.5 million candidates\\
    2014     & \bold{All India Rank 22}, JEE Mains(II), among 150,000 candidates\\
    2011     & \bold{Maharashtra Talent Search} (Govt of Maharashtra) scholarship awardee\\
  \end{tabular}
}\\
\normalfont

  \sectionspace


  \section{Experience and Projects}

  \runsubsection{}
  \descriptnonewline{Undergraduate Researcher}\location{| July'2016-Present}
  \location{Transductive learning for Derivational Morphology}
  \vspace{\topsep}
  \begin{tightitemize}
  \item Analysis of Vruddhi dataset in Sanskrit under Prof. Pawan Goyal, IIT Kharagpur.
  \item	Incorporated MAD and adsorption algorithms via junto , an open source library.
  \end{tightitemize}

  \sectionspace

  \runsubsection{}
  \descriptnonewline{Undergraduate Researcher}\location{| May'15-July'15}
  \location{Kharagpur RoboSoccer Students' Group, IIT Kharagpur}
  \begin{tightitemize}
  \item Created a path planner for differential drive autonomous robots.
  \item	Decreased slip parameter to less than 1\%.
  \item The team won Bronze at FIRA 2015, South Korea.
  \end{tightitemize}

  \sectionspace
  
  \runsubsection{}
  \descriptnonewline{Undergraduate Researcher}\location{| Aug'16-Present}
  \location{Smart Wireless Applications and Networking Group, IIT Kharagpur}
  \begin{tightitemize}
  \item The group, under the School of Information Technology IIT Kharagpur, works in the field of IoT and nano networks.
  \item	Working on bacterial nano networks and its applications in the world of IoT.
  \end{tightitemize}

  \sectionspace

  \runsubsection{}
  \descriptnonewline{Software Developer}\location{OpenSoft, IIT Kharagpur | Spring '16}
  \begin{tightitemize}
  \item Created a software to extract details from a plot and tabulate them using OpenCV and tessaract.
  \end{tightitemize}

  \sectionspace

  \runsubsection{}
  \descriptnonewline{Back End Developer}\location{Game Theory Society,
    IIT Kharagpur}
  \begin{tightitemize}
  \item Contributed to back end of an online game made using CodeIgnitor framework in php.
  \item Played by over 1000 participants from across India.
  \end{tightitemize}


  \sectionspace

  \runsubsection{}
  \descriptnonewline{Image Processing Mentor}\location{IIT Kharagpur | Dec'15}
  \begin{tightitemize}
  \item Mentored over 50 students in IEEE certified workshop.
  \item Taught various image processing techniques and algorithms.
  \end{tightitemize}

  \sectionspace

  \runsubsection{}
  \descriptnonewline{Code Club}\location{Programming Mentor | IIT Kharagpur}
  \begin{tightitemize}
  \item Problem setter and tester in coding contests at IIT Kharagpur.
  \item Mentored over 100 students in algorithms and data structures.
  \item Provided logistics support for Bitwise 2016, which saw the participation of over 1000 programmers across the globe.
  \end{tightitemize}
  
  \sectionspace

  \runsubsection{}
  \descriptnonewline{Grid Follower}\location{Robotics Workshop, IIT Kharagpur | Dec'14}
  \begin{tightitemize}
  \item Attended Texas Instruments certified workshop and built an autonomous grid follower robot using Atmel Studio.
  \item Programmed the algorithm using DFS traversal to cover the entire grid.
  \end{tightitemize}

  \section{Miscellaneous}
  \vspace{\topsep}
  \begin{tightitemize}
  \item Participated in Open IIT Data Analytics in the year 2015.
  \item Participated in Open IIT Hindi as well as English Elocution in the year 2015.
  \item National Cadet Corps cadet for 2 years , July'14 - Mar' 16
  \end{tightitemize}
\end{minipage}

\end{document}
%%% Local Variables:
%%% mode: latex
%%% TeX-master: t
%%% TeX-engine: xetex
%%% End: