%%%%%%%%%%%%%%%%%%%%%%%%%%%%%%%%%%%%%%% 
% Deedy CV/Resume
% XeLaTeX Template
% Version 1.0 (5/5/2014)
% 
% This template has been downloaded from:
% http://www.LaTeXTemplates.com
% 
% Original author:
% Debarghya Das (http://www.debarghyadas.com)
% With extensive modifications by:
% Vel (vel@latextemplates.com)
% 
% License:
% CC BY-NC-SA 3.0 (http://creativecommons.org/licenses/by-nc-sa/3.0/)
% 
% Important notes:
% This template needs to be compiled with XeLaTeX.
% 
%%%%%%%%%%%%%%%%%%%%%%%%%%%%%%%%%%%%%% 

\documentclass[a4paper]{deedy-resume} % Use US Letter paper, change to a4paper for A4 

\newcommand{\onepage}[1]{
  \ifdefined\onep
  #1
  \fi
}

\begin{document}
\namesection{Kaustubh}{Hiware}{
  Computer Science and Engineering, IIT Kharagpur \\
  --\\
  \href{mailto:hiwarekaustubh@iitkgp.ac.in}{ hiwarekaustubh@iitkgp.ac.in}  |  (+91) 8768829567\\
  E-410,Azad Hall of Residence, Indian Institute of Technology Kharagpur - 721302}
\begin{minipage}[t]{0.33\textwidth} % The left column takes up 33% of the text width of the page

  \section{Education} 

  \descriptbig{IIT Kharagpur}

  \descript{B.Tech+M.Tech in \\Computer Science and Engineering}
  \location{2014-Present | CGPA: \textit{8.18/10}}

  \sectionspace
  \sectionspace
  
  \descriptbig{Shivaji Science College, \\ Nagpur(State Board)}
  \descript{Senior Secondary Grade 12}
  \location{Overall: \textbf{95.54\%}}

  \sectionspace
  \sectionspace
  
  \descriptbig{Sandipani School, \\ Nagpur (CBSE)}
  \descript{Secondary Grade 10}
  \location{Overall: \textbf{96.8\%}}    
    
  \sectionspace

  \section{Links} 

  Github:// \href{https://github.com/kaustubhhiware}{\bf kaustubhhiware} \\
  LinkedIn:// \href{https://www.linkedin.com/in/kaustubhhiware}{\bf kaustubhhiware} \\
  Student webpage:// \href{http://cse.iitkgp.ac.in/~hknarendra/}{\bf hknarendra} \\
  
  \sectionspace

  \section{CS Coursework}
  Algorithms 1 \& 2, Data Analytics, \\
  Software Engineering, \\
  Computer Organization, \\
  Computer Networks \& lab*, \\
  Automata Theory, Compilers, \\ 
  Discrete Structures\\
  Educational Data Analytics*, \\
  Operating Systems \& lab *,  \\
  Database Management *\\

  {\footnotesize \textit{\textbf{(* denotes ongoing courses) }}} \\

  \sectionspace

  \section{Skills}

  \runsubsection{}
  \descriptbig{Programming Languages}
  C/C++ \textbullet{} Java \textbullet{} Python \\
  \textbullet{} Assembly \textbullet{} Bash

  \sectionspace
  \sectionspace

  \descriptbig{Operating Systems}
  Ubuntu \textbullet{} Microsoft Windows

  \sectionspace
  \sectionspace

  \descriptbig{Utilities \& Frameworks}
  Git \textbullet{} Django  \\
  \textbullet{} OpenCV \textbullet{} Gazebo

  \sectionspace
  \sectionspace

  \section{Interests}
  Algorithms \\
  Data Analytics \\
  Machine Learning \\
  Natural Language Processing \\
  Open source \\
\end{minipage}
\hfill
\begin{minipage}[t]{0.66\textwidth}

  \section{Awards and Achievements}

\fontspec[Path = fonts/raleway/]{Raleway-Light}\fontsize{11pt}{11pt}\selectfont {

  \begin{tabular}{rll}
    2014	 & \bold{All India Rank 1874}, JEE Advanced, among 150,000 candidates\\
    2014	 & \bold{All India Rank 277}, JEE Mains(I), among 1.5 million candidates\\
    2014     & \bold{All India Rank 22}, JEE Mains(II), among 150,000 candidates\\
    2011     & \bold{Maharashtra Talent Search}(Govt of Maharashtra)scholarship recipient\\
  \end{tabular}
}\\
\normalfont

  \sectionspace


  \section{Experience and Projects}

  \runsubsection{}
  \descriptnonewline{Undergraduate Researcher}\location{| July'2016-Present}
  \location{Transductive learning for Derivational Morphology}
  \vspace{\topsep}
  \begin{tightitemize}
  \item Analysis of Vruddhi dataset in Sanskrit under Prof. Pawan Goyal, IIT Kharagpur.
  \item	Incorporated MAD and adsorption algorithms via junto , an open source library.
  \end{tightitemize}

  \sectionspace

  \runsubsection{}
  \descriptnonewline{Software Developer}\location{| Spring '16}
  \location{OpenSoft, IIT Kharagpur }
  \begin{tightitemize}
  \item Created a module to extract graph information and tabulate using OpenCV libraries and Tesseract.
  \item The code for the same is available on \href{https://github.com/Azad-Hall/open-soft-2015-2016}{\bf GitHub.}
  \end{tightitemize}

  \sectionspace
  
  \runsubsection{}
  \descriptnonewline{Front-end Developer}\location{| May'16}
  \location{c0derunR}
  \begin{tightitemize}
  \item A Django(python) webapp for Integrated Development Environment using HackerEarth API.[WIP]
  \item Includes a static website generator which allows you to view the changes you make on the go.
  \item Open sourced the project . Available \href{https://github.com/kaustubhhiware/Code.Play}{\bf here.}
  \end{tightitemize}

  \sectionspace
  
  \runsubsection{}
  \descriptnonewline{Software Developer}\location{| Mar'16}
  \location{Department Information System}
  \begin{tightitemize}
  \item Software project in Java under Prof. Partha Pratim Das, IIT Kharagpur .
  \item Created a software allowing the administrator to perform managerial tasks. Available on \href{https://github.com/kaustubhhiware/DepInfosys}{\bf GitHub.}
  \end{tightitemize}

  \sectionspace

  \runsubsection{}
  \descriptnonewline{Tiny-c compiler}\location{| Nov'16}
  \location{cOMPILER}
  \begin{tightitemize}
  \item A Compilers course project under Prof. Partha Pratim Das, IIT Kharagpur.
  \item A restricted grammar C compiler made using flex,yacc and gnu assembler. Available on \href{https://github.com/kaustubhhiware/cOMPILER}{\bf GitHub.}
  \end{tightitemize}
  
  \sectionspace

  \runsubsection{}
  \descriptnonewline{Software Team Member}\location{| Mar'15-Nov'15}
  \location{Autonomous Ground Vehicle, IIT Kharagpur}
  \begin{tightitemize}
  \item	Worked as a member of Stimulation team for real-environment testing of image detection algorithms.
  \item Simulation using ROS, Gazebo framework.
 \end{tightitemize}


  \section{Miscellaneous}
  \vspace{\topsep}
  \begin{tightitemize}
  \item Frequent open source contributor on \href{https://github.com/kaustubhhiware}{\bf GitHub.}
  \item Participated in Open IIT Data Analytics in the year 2015.
  \item Participated in Open IIT Hindi as well as English Elocution in the year 2015.
  \item National Cadet Corps cadet for 2 years, July'14 - Mar' 16  
  \end{tightitemize}
\end{minipage}

\end{document}
%%% Local Variables:
%%% mode: latex
%%% TeX-master: t
%%% TeX-engine: xetex
%%% End: