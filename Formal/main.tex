% Important note:
% This template requires the resume.cls file to be in the same directory as the
% .tex file. The resume.cls file provides the resume style used for structuring the
% document.
%
%%%%%%%%%%%%%%%%%%%%%%%%%%%%%%%%%%%%%%%%%

%----------------------------------------------------------------------------------------
%	PACKAGES AND OTHER DOCUMENT CONFIGURATIONS
%----------------------------------------------------------------------------------------

\documentclass{resume} % Use the custom resume.cls style

\usepackage[left=0.3in,top=0.6in,right=0.3in,bottom=0.6in]{geometry} % Document margins
\usepackage{parnotes}
\usepackage{hyperref}
\usepackage{xcolor}
\usepackage{array}

\definecolor{mypurple}{HTML}{794BC4}

\hypersetup{
    colorlinks=true,
    linkcolor=purple,
    filecolor=magenta,      
    urlcolor=mypurple,
}


\name{Kaustubh   Hiware} % Your name
%\address{ C-123 Azad Hall of Residence, IIT Kharagpur, West Midnapore \\ West Bengal, India 721302} % Your address
% \address{150, Ulhasnagar, Manewada Square, Nagpur \\ Maharashtra, India 440027} % Your secondary address (optional)
\address{\href{mailto:hiwarekaustubh@gmail.com}{ hiwarekaustubh@gmail.com} \\ +91 8668392709 \\ \url{kaustubhhiware.in}} % Your phone number and email
% \address{\url{kaustubhhiware.in}}
\begin{document}

%----------------------------------------------------------------------------------------
%	EDUCATION SECTION
%----------------------------------------------------------------------------------------

% \vspace{-1.5em}
\begin{rSection}{Education}

{\bf \large Indian Institute of Technology Kharagpur, West Bengal} \hfill \textbf{Jul $^{\prime}$14 - Jul $^{\prime}$19} \\ 
B.Tech. \& M.Tech. Hons. in Computer Science \& Engineering \hfill CGPA: 8.47/10\\
\vspace{-1.5em}

{\bf \large Shivaji Science College, Nagpur} \hfill \textbf{2014} \\ 
Senior Higher Secondary School Education, Maharashtra State Board \hfill  Percentile: 99.98\% \\
\vspace{-1.5em}

{\bf \large Sandipani School, Nagpur} \hfill \textbf{2012} \\ 
Higher Secondary School Education, Central Board for Secondary Education \hfill CGPA: 10/10\\
\vspace{-1.5em}

\end{rSection}

%----------------------------------------------------------------------------------------
%	Submitted Papers SECTION
%----------------------------------------------------------------------------------------
\begin{rSection}{Accepted Publications}
$\bullet$ "\href{https://www.sciencedirect.com/science/article/abs/pii/S1071581918304889}{Emotion Detection from Touch Interactions during Text Entry on Smartphones}" \\
- Surjya Ghosh, \textbf{Kaustubh Hiware}, Niloy Ganguly, Bivas Mitra, Pradipta De,\\
in: \textit{International Journal of Human-Computer Studies, 2019.} \\
\textbf{\small Key Topics}: Emotion detection, Smartphone, Human Computer Interaction.

$\bullet$ "\href{https://dl.acm.org/authorize.cfm?key=N677554}{Does Emotion Influence the Use of Auto-suggest during Smartphone Typing?}" \\
- Surjya Ghosh, \textbf{Kaustubh Hiware}, Bivas Mitra, Niloy Ganguly, Pradipta De,\\
in: \textit{International Conference on Intelligent User Interfaces, ACM, IUI 2019.} \\
\textbf{\small Key Topics}: Emotion detection, Smartphone, Human Computer Interaction.

$\bullet$ "\href{https://www2018.thewebconf.org/proceedings/#wrk-131}{SAVITR: A System for Real-time Location Extraction from Microblogs during Emergencies.}" \\
- Ritam Dutt, \textbf{Kaustubh Hiware}, Avijit Ghosh, Rameshwar Bhaskaran, \\
in: \textit{Exploitation of Social Media for Emergency Relief and Preparedness (SMERP), colocated with The Web Conference 2018 (formerly WWW)}\\
\textbf{\small Key Topics}: Emergencies, microblogs, location extraction, Social Computing.

$\bullet$ "\href{https://ecir2019.org/accepted-papers/#reproducibilitypapers}{A Comparative Study of Summarization Algorithms applied to Legal Case Judgments}" \\
- Paheli Bhattacharya, \textbf{Kaustubh Hiware}, Subham Rajgaria, Nilay Pochhi, Kripabandhu Ghosh, Saptarshi Ghosh,\\
in: \textit{European Conference on Information Retrieval, ECIR 2019} \\
\textbf{\small Key Topics}: Summarization, Reproducibility, Legal Document, Supervised, Natural Language Processing.

$\bullet$ "\href{http://www.aclweb.org/anthology/W17-2409}{A Graph Based Semi-Supervised Approach for Analysis of Derivational Nouns in Sanskrit}" \\
- Amrith Krishna, P. Satuluri, H. Ponnada, M. Ahmed, G. Arora, \textbf{Kaustubh Hiware}, Pawan Goyal, \\
in: \textit{TextGraphs-11: the Workshop on Graph-based Methods for Natural Language Processing, ACL 2017}\\
\textbf{\small Key Topics}: Sanskrit Word-Segmentation, Network-based learning, Natural Language Processing

\end{rSection}

%----------------------------------------------------------------------------------------
%	ACADEMIC PROJECTS SECTION
%----------------------------------------------------------------------------------------

\begin{rSection}{Academic Projects}

\begin{rSubsection}{ \large Deep Learning in Legal Data Analytics}{\textbf{\large July $^{\prime}$18 - May$^{\prime}$19}}{Masters' Thesis Project under Prof. Saptarshi Ghosh}{CSE Dept. IIT Kharagpur}
% \begin{rSubsection}{ \large Deep Learning in Legal Data Analytics}{\textbf{\large July $^{\prime}$18 - Present}}{}{}
\item \textit{Abstract} - Utilizing legal knowledge to deploy a \textbf{search engine} with thematic segmentation.
\item Analyzed summarization techniques for Indian legal documents.A paper has been accepted on the same at \href{http://ecir2019.org}{ECIR.}
\item Applied catchphrase detection to present information in a concise manner on the proposed search engine. A paper was submitted at \href{http://sigir.org/sigir2019/}{SIGIR, 2019.}
\end{rSubsection}
%------------------------------------------------

% \vspace{-0.5em}
\begin{rSubsection}{ \large Emotion detection using smart devices}{\textbf{\large July $^{\prime}$17 - May $^{\prime}$19}}{Bachelor's Thesis Project under Prof. Bivas Mitra and Prof. Niloy Ganguly}{CSE Dept. IIT Kharagpur}
% \begin{rSubsection}{ \large Emotion detection using smart devices}{\textbf{\large July $^{\prime}$17 - Present}}{}{}
\item \textit{Abstract} - We investigate the role of both types of touch interactions in \textbf{multi-state emotion detection} for a generic context. Emotions were recorded via a specifically developed android application.

\item Involves jointly modeling typing interactions and correlate them with user provided self-reports to build a \textbf{personalized machine learning model}, to obtain an accuracy of \textbf{73\%} for 22 participants.

\item 2 research publications accepted - at \href{https://iui.acm.org/2019}{IUI, 2019} and \href{https://www.journals.elsevier.com/international-journal-of-human-computer-studies}{IJHCS} journal.
\end{rSubsection}
%------------------------------------------------

% \vspace{-0.5em}
\begin{rSubsection}{ \large NARMADA: Need and Availability Resource Managing Assistant for Disasters and Adversities: \href{http://narmada.herokuapp.com}{narmada.herokuapp.com}}{\textbf{\large Aug $^{\prime}$18 - Apr$^{\prime}$19}}{Research Project under Prof. Saptarshi Ghosh}{CSE Dept. IIT Kharagpur}
% \begin{rSubsection}{ \large Resource Matching Assistant for Disasters: \href{http://narmada.herokuapp.com}{narmada.herokuapp.com}}{\textbf{\large Aug $^{\prime}$18 - Present}}{}{}
\item \textit{Abstract} - Create a \textbf{during-disaster system} to aid \textbf{disaster mitigation efforts}, to tackle misinformation and lack of coordination. Matches needs and availability tweets via resource embeddings with an F-score of \textbf{88\%}.
\item Prototype secured \textbf{first} place in Microsoft's CodefunDo++ hackathon out of 242 participants, one of the 21 teams from all over India selected to showcase their product on a \textbf{national level}.
\item Demo paper was submitted at \href{http://sigir.org/sigir2019/}{SIGIR, 2019.}
\end{rSubsection}
%------------------------------------------------


% \vspace{-0.5em}
\begin{rSubsection}{ \large SAVITR: A System for Real-time Location Extraction from Microblogs during Emergencies: \href{http://savitr.herokuapp.com}{savitr.herokuapp.com}}{\textbf{\large July $^{\prime}$17 - April $^{\prime}$18}}{Research Project under Prof. Saptarshi Ghosh}{CSE Dept. IIT Kharagpur}
% \begin{rSubsection}{ \large Real-time Location Extraction from Microblogs: \href{http://savitr.herokuapp.com}{savitr.herokuapp.com}}{\textbf{\large July $^{\prime}$16-April $^{\prime}$17}}{}{}
\item The objective was to create a \textbf{pre-warning system} to detect potential disaster situations. 
\item Helps identify \textbf{geospatial distribution} of important tweets and classifying tweets as situational and non-situational, which can help in \textbf{disaster relief and mitigation}. A paper on the same was accepted at \href{https://www.cse.iitk.ac.in/users/kripa/smerp2018/}{SMERP - WWW'18}.
\end{rSubsection}
%------------------------------------------------

%%\begin{rSubsection}{ \large Transductive learning for Derivational Morphology}{\textbf{\large July $^{\prime}$16-April $^{\prime}$17}}{Research Project under Asst. Prof. Pawan Goyal Ghosh}{CSE Dept. IIT Kharagpur}
%\item \textit Developed a graph based semi supervised approach for analysis of derivative nouns in Sanskrit.
%%\item Constructed character n-grams and semantic word vectors for Sanskrit dictionary. 
%%\item Proposed model is effective even when labelled dataset is only 5\% to that of the entire dataset.
%%\end{rSubsection}
%------------------------------------------------

% \begin{rSubsection}{ \large Understanding code-mixing patterns in celebrity tweets}{\textbf{\large Aug $^{\prime}$17 - Dec $^{\prime}$17}}{Academic Project under Prof. Niloy Ganguly}{CSE Dept. IIT Kharagpur}

% \item \textit{Abstract} -Distinguish \textbf{code-mixing} and code-borrowing instances in the context of celebrity tweets. We try to understand if a word from a foreign language is used by a celebrity, how likely is it to be borrowed by their followers. 
% \item Analyzed \textbf{Sense deviation} when words from foreign languages used in Social Media context. Developed \textbf{word embeddings} for social media dataset.
% \item Transformed word embeddings into another vector space using \textbf{Hamilton's Diachronic method}, compared similarity of senses and transformed vectors.
% \end{rSubsection}
%------------------------------------------------

% \vspace{-0.5em}
\begin{rSubsection}{ \large Adversarial attacks on DNN Classifiers}{\textbf{\large March $^{\prime}$18}}{Academic Project under Prof. Sudeshna Sarkar}{CSE Dept. IIT Kharagpur}
% \begin{rSubsection}{ \large Adversarial attacks on DNN Classifiers}{\textbf{\large March $^{\prime}$18}}{}{}

\item \textit{Abstract} -Craft adversarial images using Fast Gradient Sign Method and Jacobian Based Saliency Map attacks so as to befool pre-existing DNN classifiers, while remaining correctly classified by humans. 
\item Used Fast Gradient Sign Method to reduce accuracy by 25\% in MNIST dataset, and 62\% in CIFAR dataset.
\end{rSubsection}
%------------------------------------------------

% \begin{rSubsection}{ \large Graphwise - Large Scale Distributed Graph Processing}{\textbf{\large April $^{\prime}$17}}{Academic Project under Prof. Pabitra Mitra}{CSE Dept. IIT Kharagpur}
% % \begin{rSubsection}{ \large Graphwise - Large Scale Distributed Graph Processing}{\textbf{\large April $^{\prime}$17}}{}{}

% \item Used Louvain algorithm to detect \textbf{communities in large networks} with scala and t-SNE, a dimensionality reduction technique for data visualisation in python.

% \item Reached a maximum modularity of \textbf{98.6\%}. Open sourced the project on \href{https://github.com/kaustubhhiware/Graphwise}{Github}.
% \end{rSubsection}

% \begin{rSubsection}{ \large Core CS projects}{}{}{CSE Dept. IIT Kharagpur}

% \item Implemented a queue based scheduler, memory management and signal handling for \href{https://github.com/kaustubhhiware/OSLab}{pintOS prototype}.
% \item Implemented a \href{https://github.com/kaustubhhiware/cOMPILER}{restricted grammar C compiler} made using flex,yacc and gnu assembler.
% \item Implemented L2 forwarding, L3 static routing and reliable UDP protocols, and P2P file-sharing system.
% \end{rSubsection}

%------------------------------------------------


\end{rSection}


%----------------------------------------------------------------------------------------
%	INTERNSHIPS SECTION
%----------------------------------------------------------------------------------------

\begin{rSection}{Internships}
\begin{rSubsection}{\large \href{https://drive.google.com/file/d/1EcfIC_g8bx3whaxMdOkFyZx4TCqA1RAq/view?usp=sharing}{Schlumberger, Pune Technology Center}}{\textbf{\large May $^{\prime}$18- Jul $^{\prime}$18}}{Software Intern}{Pune, India}
\item Developed {\bf RESTful backend API} from scratch in C\#, .NET that made existing computation engine accessible.
\item Wrote a DLS-compliant {\bf frontend in Angular}, which was used  by multiple field experts from South-east Asia.
\item Rewarded a {\bf Pre-Placement Offer} as Software Engineer for diligent contribution to the organisation.
\end{rSubsection}

\begin{rSubsection}{\large \href{https://drive.google.com/file/d/0B5iU6cWw36rOVWZIcllPY3RSd2c/view?usp=sharing}{Elanic}}{\textbf{\large May $^{\prime}$17- Jul $^{\prime}$17}}{Backend Developer Intern}{Bangalore, India}
\item Built complete \textbf{CRUD} operations for 2 classes. Abstraction written extended to 50\% of all classes.
\item Indexed 30M items, performing exhaustive data sanitization. Automated testing suite using Jenkins.
\item Developed a npm module \href{https://www.npmjs.com/package/urlparamify}{urlparamify} to be used in website, which has received over \textbf{3000+} downloads so far.
\end{rSubsection}

%------------------------------------------------

\end{rSection}

%------------------------------------------------


%----------------------------------------------------------------------------------------
%	OPEN SOURCE PROJECTS SECTION
%----------------------------------------------------------------------------------------


\begin{rSection}{\large Open-Source Projects}{}{I like to work on hobby projects and opensource them. More can be found at \textbf{\href{https://github.com/kaustubhhiware}{github.com/kaustubhhiware}}}

{\bf \large \href{https://github.com/kaustubhhiware/facebook-archive}{facebook-archive}} \hfill \textbf{March $^{\prime}$18} \\ 
Facebook data analysis toolkit, written in Python. Plot trends and observe patterns in usage.
%----------------------------------------------

{\bf \large Foodspark - \url{foodkgp.pythonanywhere.com}} \hfill \textbf{Feb $^{\prime}$17} \\ 
Academic Project under Prof. Pabitra Mitra. A Django(python) webapp for \textbf{online food ordering system}.
%----------------------------------------------

{\bf \large c0derunR - \url{c0derunr.herokuapp.com}} \hfill \textbf{Jan $^{\prime}$17} \\ 
An \textbf{NLP-powered Integrated Development Environment} using HackerEarth API. Offers a static website generator reflecting on-the-go changes.
%----------------------------------------------

{\bf \large \href{https://github.com/kaustubhhiware/NotiFyre}{NotiFyre}} \hfill \textbf{Dec $^{\prime}$16} \\ 
A \textbf{terminal task notifier} with support of basic customizations. Supports bash, zsh and fish shell.
%----------------------------------------------

% {\bf \large \href{https://github.com/kaustubhhiware/cOMPILER}{Tiny-C compiler}} \hfill \textbf{Nov $^{\prime}$16} \\ 
% A restricted grammar C compiler made using flex,yacc and gnu assembler. A Compilers course project.
%----------------------------------------------

% {\bf \large Wordabulary} \hfill \textbf{May $^{\prime}$16} \\ 
% A python pip module for grammar statistics. Support trivial document analysis. Developed as a module to validate Zipf’s law -  \href{https://github.com/kaustubhhiware/Wordabulary}{GitHub}.
%----------------------------------------------

% {\bf \large Data extraction from a 2-D plot} \hfill \textbf{Feb $^{\prime}$16 - Mar $^{\prime}$16} \\ 
% Developed during Inter-hall OpenSoft competition. Created a module to extract graph information from a plot and create the corresponding data table using OpenCV libraries and Tesseract -\href{https://github.com/Azad-Hall/open-soft-2015-2016}{GitHub}.
%----------------------------------------------

% {\bf \large Department Information System} \hfill \textbf{Mar $^{\prime}$16} \\ 
% Created a software which allows the administrator to perform managerial tasks like resource management, course handling and student graduing. Source code on \href{https://github.com/kaustubhhiware/DepInfosys}{GitHub}.
%----------------------------------------------

%------------------------------------------------
\end{rSection}


%----------------------------------------------------------------------------------------
%	ACADEMIC ACHIEVEMENTS SECTION
%----------------------------------------------------------------------------------------

\begin{rSection}{Academic Achievements}

{\bf \large Joint Entrance Examination (JEE) - Advanced} \hfill \textbf{2014} \\ 
Among top \textbf{1.2\%} of the \textbf{0.15 million} candidates, ranking \textbf{1874}.

{\bf \large Joint Entrance Examination (JEE) - Mains(B.E)} \hfill \textbf{2014} \\ 
Among top \textbf{0.01\%} of the \textbf{1.4 million} candidates, ranking \textbf{277}.

{\bf \large Joint Entrance Examination (JEE) - Mains(B.Arch)} \hfill \textbf{2014} \\ 
Among top \textbf{0.01\%} of the \textbf{0.15 million} candidates, ranking \textbf{22}.

\end{rSection}

%----------------------------------------------------------------------------------------
%	RELEVANT COURSES SECTION
%----------------------------------------------------------------------------------------

\begin{rSection}{Relevant Courses}
\begin{tabular}{p{3.7cm} p{3.5cm} p{5.7cm} p{4.1cm}}
Deep Learning$^+$ & Machine Learning & Natural Language Processing$^+$ & Image Processing$^+$ \\
Data Analytics$^+$ & Social Computing$^+$ & Database Management Systems$^+$ & Computer Networks$^+$  \\
Artificial Intelligence & Algorithms 1$^+$ \& 2 & Software Engineering$^+$ & Operating Systems$^+$ \\
Distributed Systems$^+$ & Compilers$^+$ & Parallel \& Distributed Algorithms & Computer Organization\\
\end{tabular}\\
% Distributed systems, Parallel and Distributed Algo, Compilers, COA
\centerline{\footnotesize \textit{ ( $^+$ denotes practicum included )  }}
% \centerline\footnotesize \textit{(* denotes ongoing courses, $^+$ denotes practicum included)\\
% \begin{center}
% \begin{tabular}{p{3.9cm} p{3.0cm} p{6.5cm} p{4.0cm}}
%   Deep Learning$^+$ & Machine Learning & Natural Language Processing$^+$ & Artificial Intelligence\\
%   Social Computing$^+$ & Data Analytics$^+$ & Database Management$^+$ & Operating Systems$^+$\\
%   Computer Networks$^+$ & Compilers$^+$ & Computer Organisation$^+$ & Matrix Algebra\\
%   Software Engineering$^+$ & Algorithms$^+$ & Formal Language \& Automata Theory & Discrete Structures\\
% \end{tabular}
% \footnotesize \textit{(* denotes ongoing courses, $^+$ denotes practicum included)}
% \end{center}
\end{rSection}

%----------------------------------------------------------------------------------------
%	TECHNICAL SKILLS SECTION
%----------------------------------------------------------------------------------------

\begin{rSection}{Technical Skills}

\begin{tabular}{ @{} >{\bfseries}l @{\hspace{4ex}} l }
Programming Languages & Python $\bullet$ C $\bullet$ C++ $\bullet$ JavaScript $\bullet$ Java $\bullet$ Assembly $\bullet$ Scala $\bullet$ R \\
Web & HTML $\bullet$ CSS / SCSS $\bullet$ Django $\bullet$ Jekyll $\bullet$ Bootstrap $\bullet$ React $\bullet$ Angular \\
Tools \& Frameworks & Git $\bullet$ NodeJS $\bullet$ Shell/Fish $\bullet$ Elasticsearch $\bullet$ TravisCI $\bullet$ Heroku $\bullet$ \LaTeX \\
%  Tools \& Frameworks & Git $\bullet$ NodeJS $\bullet$ Shell $\bullet$ Elasticsearch $\bullet$ MongoDB $\bullet$ MySQL $\bullet$ \LaTeX \\
Databases & MongoDB $\bullet$ MySQL $\bullet$ SQLite3 \\
% Operating Systems & Ubuntu/Linux $\bullet$ Windows
\end{tabular}

\end{rSection}

%----------------------------------------------------------------------------------------
%	POSITIONS OF RESPONSIBILITES SECTION
%----------------------------------------------------------------------------------------

\begin{rSection}{Positions of Responsibility}

\begin{rSubsection}{\large MetaKGP}{\textbf{\large Oct $^{\prime}$17 - Jan $^{\prime}$19}}{Maintainer}{IIT Kharagpur, WB}
\item Metakgp is a loose association of engineers, hackers, artists, and students from IIT Kharagpur, who collaborate on various technical and non-technical projects.
% \item Metakgp is a loose association of students who collaborate on various technical and non-technical projects.
\item Managing development of multiple open source projects, with individual userbases of over 5000 students.
\item Organizing \href{https://www.youtube.com/watch?v=srH_yJJFK80&list=PLxBVN59ffbfKmADdIpkNCs4jfxJ-Nyhqx}{Demo Days}, as a way to incentivize working on hobby projects.
\end{rSubsection}
%------------------------------------------------


% \begin{rSubsection}{\large coala: Language Independent Code Analysis}{\textbf{\large Mar $^{\prime}$17 - Present}}{Open source Developer}{}
% \item coala provides a unified command-line interface for linting and fixing code, independent of programming languages used.
% \item Sucessfully merged 4 out of 6 submitted patches so far.
% \end{rSubsection}
%------------------------------------------------

% \begin{rSubsection}{\large Teaching Assistant, Machine Learning}{\textbf{\large Jul $^{\prime}$18 - Nov $^{\prime}$18}}{}{}
% \item Assist in evaluating assignments of 220+ enrolled students, the \textbf{largest post-graduate} course.
% \end{rSubsection}

% \begin{rSubsection}{\large Teaching Assistant, Principles of Programming Languages}{\textbf{\large Jan $^{\prime}$19 - Present}}{}{}
% \item Assist in evaluating, setting assignments for 120 enrolled students.
% \end{rSubsection}
%------------------------------------------------

\end{rSection}

%----------------------------------------------------------------------------------------
%	ACHIEVEMENTS SECTION
%----------------------------------------------------------------------------------------

\begin{rSection}{Co-Curricular Achievements}

$\bullet$ Received \href{https://sites.nyuad.nyu.edu/hackathon/}{Honorable Mention} in IIT Kharagpur Gymkhana Technology Awards. (Apr 2019) \\
$\bullet$ Secured \href{https://sites.nyuad.nyu.edu/hackathon/}{Second Place} in NYU Abu Dhabi's International Hackathon on Social Good. (Apr 2019) \\
$\bullet$ Part of Bronze winning contingent in 7th Inter IIT Tech Meet, 2018 held at IIT Bombay. \\
$\bullet$ Mentored a team of 20 to secure \href{}{Silver} in Inter Hall OpenSoft competition, IIT Kharagpur (March 2019).\\
$\bullet$ Secured \href{https://drive.google.com/file/d/13SDYiXpSi7zXI_bZTb_tj8YukIONf9vn/view?usp=sharing}{First Prize} in Oral Presentation at Research Conclave, IIT Guwahati. (March 2019)\\
$\bullet$ Received \href{https://drive.google.com/file/d/1qDoD5THDmMyWmFPaEpmMYpaEBe0T35ay/view?usp=sharing}{First Prize} for  Research  in the 2nd Workshop on Science of Happiness, IIT Kharagpur. (Apr 2019) \\
$\bullet$ Proposal \href{https://docs.google.com/presentation/d/1ndJbIQCxFEf__I55yOOME-a_UGWDv1KMlukVE2lbDeA/}{ResSearch} selected for final round of \href{https://drive.google.com/open?id=0B5iU6cWw36rOamZLWHZENWdlY0k}{Smart India Hackathon} amongst 8000 entries. (Mar 2017) \\
$\bullet$ Shortlisted in \href{https://www.aicte-india.org/sites/default/files/Shortlisted and waitlisted Teams indo singapore hackathon 2018.pdf}{top 24} teams to represent India in first \href{http://www.ntu.edu.sg/events/events/Pages/Singapore-India-Hackathon-20181002-2080.aspx}{Indo-Singapore Hackathon}.\\
% $\bullet$ Received Amul Vidya Shree Award, for outstanding performance in Class 10 board exams.\\
$\bullet$ Qualified for Innovation in Science Pursuit for Inspired Research {\bf \large (INSPIRE)} scholarship, awarded to only \textbf{0.01\%} Indian students every year. \\
% $\bullet$ Qualified for INSPIRE scholarship, awarded to only \textbf{0.01\%} students every year. \\
$\bullet$ Received scholarship under Maharashtra Talent Search Examination, obtaining a percentile of 96\%.
%$\bullet$ School-level representation at multiple quiz competitions, debate and extempores.
%$\bullet$ Have consistently been a member of school student board, holding the positions of Head Boy and Deputy Head Boy consecutively.

\vspace{2em}
% \vspace{-0.9em}
\begin{rSubsection}{\large Speaking Experience}{}{}{}

\item \href{https://www.youtube.com/watch?v=lk4ciY3NSbA}{Innovations, Trends and Licensing in Open Source} - organized by Kharagpur Open Source Society, at Kshitij, IIT Kharagpur's \href{https://www.facebook.com/events/1387262571380225/}{Open Source Summit}. January, 2018.
\item \href{https://twitter.com/IIITAPragma/status/1085833015995953152}{You should be doing Research right now} - organized by GeekHaven, at \href{https://pragmaconf.tech}{Pragma}, IIIT Allahabad. January, 2019.

\end{rSubsection}

% \vspace{-0.9em}
\begin{rSubsection}{\large Teaching Experience}{}{}{}

\item Teaching Assistant for the course Machine Learning under Prof. Pabitra Mitra, Autumn 2018. Assisted in evaluating assignments of 220+ enrolled students, the \textbf{largest post-graduate} course that semester.
\item Teaching Assistant for Principles of Programming Languages under Prof. Partha Pratim Das, Spring 2019. Assisted in evaluating and setting assignments for 120 students.
\item Helped organize Introductory classes for Python (for first years) with Kharagpur Open Source Society. (July 2017)
\end{rSubsection}

% \vspace{-0.9em}
\begin{rSubsection}{\large Presenting Experience}{}{}{}

\item (Poster) ICNDE, 1st International Conference on the Networked Digital Earth, IIT Kharagpur, March 2018.
\item (Poster) Engineers' Conclave, 7th Inter IIT Tech Meet, IIT Bombay, December 2018.

\end{rSubsection}

%----------------------------------------------------------------------------------------

%----------------------------------------------------------------------------------------
%	VOLUNTEER EXPERIENCES SECTION
%----------------------------------------------------------------------------------------

%\begin{rSection}{Volunteer Experiences}

% \begin{rSubsection}{\large Volunteer Experience}{}{}{}
% \item 
% \end{rSubsection}
\vspace{-0.9em}
{\large {\textbf{Volunteer Experience}}

% \href{https://kwoc.kossiitkgp.in/}{Kharagpur Winter of Code} , \href{https://gssoc.tech/projects.html#facebook-archive}{GirlScript Summer of Code}
\begin{rSubsection}{\large \href{https://codein.withgoogle.com/}{Google code-in}}{\textbf{ Oct $^{\prime}$18 - Dec $^{\prime}$18}}{Open Source Mentor}{}
\item Guided pre-university students (ages 13-17) to open source software development, on multiple open sourced projects under the organization \href{http://coala.io}{coala}.
\end{rSubsection}
%---------------------------------------------------------------------

\vspace{-0.5em}
\begin{rSubsection}{\large \href{https://kwoc.kossiitkgp.in/}{Kharagpur Winter of Code}, \href{https://gssoc.tech/projects.html\#facebook-archive}{GirlScript Summer of Code}}{\textbf{ Dec $^{\prime}$17, May $^{\prime}$18 - Aug $^{\prime}$18}}{Open Source Mentor}{IIT Kharagpur / India}
\item Guided undergraduate freshers towards contributing to open source, on multiple personal projects.
\end{rSubsection}
%---------------------------------------------------------------------

\vspace{-0.5em}
\begin{rSubsection}{\large {\color{mypurple} Student Welfare Group (SWG)}}{\textbf{\large Jul $^{\prime}$16 - Present}}{Mentor}{IIT Kharagpur}
\item Providing guidance to a group of 6 Undergraduate Students under SWG's \textit{Student-Mentor Program} to smoothen their academic transition as well as bolster growth in non-academic dimensions
% \item Help them prepare a better study plan for upcoming semester by providing an incisive analysis of the upcoming subjects along with insightful details about diverse career opportunities to pursue
\end{rSubsection}
%---------------------------------------------------------------------

\vspace{-0.5em}
\begin{rSubsection}{\large {\color{mypurple} 3 Bengal Tech Air Squadron}}{\textbf{\large Jul $^{\prime}$14 - Jul $^{\prime}$16}}{Cadet}{IIT Kharagpur, WB}
\item Actively participated in numerous \textit{Tree Plantation \& Blood Donation Camps}
\item Attended the Annual NCC Camp \& organized as well as presented at the closing ceremony.
\end{rSubsection}
%---------------------------------------------------------------------


% \begin{rSubsection}{\large Gopali Youth Welfare Society}{\textbf{\large Aug $^{\prime}$14 - Apr $^{\prime}$15}}{Junior Executive}{IIT Kharagpur, WB}
% \item Actively participated in numerous \textit{Tree Plantation \& Blood Donation Camps}
% \item Attended the Annual NCC Camp \& organised as well as presented at the closing ceremony
% \end{rSubsection}
%---------------------------------------------------------------------

\end{rSection}

	\vspace*{\fill}
	\begin{flushright}
	\footnotesize Last updated: May 09, 2019.
	\end{flushright}
%----------------------------------------------------------------------------------------
%----------------------------------------------------------------------------------------
\end{document}