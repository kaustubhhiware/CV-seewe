% Important note:
% This template requires the resume.cls file to be in the same directory as the
% .tex file. The resume.cls file provides the resume style used for structuring the
% document.
%
%%%%%%%%%%%%%%%%%%%%%%%%%%%%%%%%%%%%%%%%%

%----------------------------------------------------------------------------------------
%	PACKAGES AND OTHER DOCUMENT CONFIGURATIONS
%----------------------------------------------------------------------------------------

\documentclass{resume} % Use the custom resume.cls style

\usepackage[left=0.75in,top=0.6in,right=0.75in,bottom=0.6in]{geometry} % Document margins
\usepackage{parnotes}
\usepackage{hyperref}

\name{ Kaustubh   Hiware} % Your name
%\address{ C-123 Azad Hall of Residence, IIT Kharagpur, West Midnapore \\ West Bengal, India 721302} % Your address
\address{150, Ulhasnagar, Manewada Square, Nagpur \\ Maharashtra, India 440027} % Your secondary address (optional)
\address{+91 8768829567 \\ hiwarekaustubh@gmail.com \\  hiwarekaustubh@iitkgp.ac.in} % Your phone number and email
\address{\url{kaustubhhiware.github.io}}
\begin{document}

%----------------------------------------------------------------------------------------
%	EDUCATION SECTION
%----------------------------------------------------------------------------------------

\begin{rSection}{Education}

{\bf \large Indian Institute of Technology Kharagpur, West Bengal} \hfill \textbf{Jul $^{\prime}$14 - Jul $^{\prime}$19}\textit{(expected)} \\ 
B.Tech. \& M.Tech. Hons. in Computer Science \& Engineering \\
Overall CGPA: 8.19/10

{\bf \large Shivaji Science College, Nagpur} \hfill \textbf{2014} \\ 
Senior Higher Secondary School Education, Maharashtra State Board \\
Overall Percentage: 95.54\%. Percentile: 99.98\%

{\bf \large Sandipani School, Nagpur} \hfill \textbf{2012} \\ 
Higher Secondary School Education, Central Board for Secondary Education \\
Overall CGPA: 10/10

\end{rSection}

%----------------------------------------------------------------------------------------
%	ACADEMIC ACHIEVEMENTS SECTION
%----------------------------------------------------------------------------------------

\begin{rSection}{Academic Achievements}

{\bf \large Joint Entrance Examination (JEE) - Advanced} \hfill \textbf{2014} \\ 
Among top \textbf{1.2\%} of the \textbf{0.15 million} candidates, ranking \textbf{1874}.

{\bf \large Joint Entrance Examination (JEE) - Mains(B.E)} \hfill \textbf{2014} \\ 
Among top \textbf{0.01\%} of the \textbf{1.4 million} candidates, ranking \textbf{277}.

{\bf \large Joint Entrance Examination (JEE) - Mains(B.Arch)} \hfill \textbf{2014} \\ 
Among top \textbf{0.01\%} of the \textbf{0.15 million} candidates, ranking \textbf{22}.

\end{rSection}

%----------------------------------------------------------------------------------------
%	Submitted Papers SECTION
%----------------------------------------------------------------------------------------
\begin{rSection}{Accepted Publications}

[a] Ritam Dutt, \textbf{Kaustubh Hiware}, Avijit Ghosh, Rameshwar Bhaskaran.\href{https://www.cse.iitk.ac.in/users/kripa/smerp2018/savitr-system-real-ritam.pdf}{SAVITR: A System for Real-time Location Extraction from Microblogs during Emergencies.} "In: \textit{Exploitation of Social Media for Emergency Relief and Preparedness (SMERP), colocated with The Web Conference 2018 (formerly WWW)}
\textbf{\small Key Topics}: Emergencies, microblogs, location extraction, Geonames

[b] Amrith Krishna, Pavankumar Satuluri, Harshavardhan Ponnada, Muneeb Ahmed, Gulab Arora, \textbf{Kaustubh Hiware}, Pawan Goyal.\href{http://www.aclweb.org/anthology/W17-2409}{A Graph Based Semi-Supervised Approach for Analysis of Derivational Nouns in Sanskrit} "In: \textit{TextGraphs-11: the Workshop on Graph-based Methods for Natural Language Processing, ACL 2017}
\textbf{\small Key Topics}: Sanskrit Word-Segmentation, Network-based learning, Natural Language Processing

\end{rSection}

%----------------------------------------------------------------------------------------
%	ACADEMIC PROJECTS SECTION
%----------------------------------------------------------------------------------------

\begin{rSection}{Academic Projects}

\begin{rSubsection}{ \large Emotion detection using smart devices}{\textbf{\large Jan $^{\prime}$17 - Present}}{Bachelor's Thesis Project under Asst. Prof. Bivas Mitra}{CSE Dept. IIT Kharagpur}
\item \textit{Abstract} - We investigate the role of both types of touch interactions in multi-state emotion detection for a generic context. We distinguish between typing and swyping by using a clustering method and extract features corresponding to individual interaction.

\item Involves jointly modeling typing and swyping features and correlate them with user provided self-reports to build a personalized machine learning model, which can can detect four emotion states (happy, sad, stressed, relaxed.

\item 2 research publications submitted pertaining to this project, pending review.
\end{rSubsection}
%------------------------------------------------

\begin{rSubsection}{ \large SAVITR: A System for Real-time Location Extraction from Microblogs during Emergencies.}{\textbf{\large July $^{\prime}$16-April $^{\prime}$17}}{Research Project under Asst. Prof. Saptarshi Ghosh}{CSE Dept. IIT Kharagpur}
\item The objective was to create a pre-warning system to detect potential disaster situations, by continually monitoring tweets in real-time. This tool helps us understand the geospatial distribution of important tweets across time.
\item Helps in detection of important tweets and classifying tweets as situational and non-situational, which can help in disaster relief and mitigation.
\end{rSubsection}
%------------------------------------------------

\begin{rSubsection}{ \large Transductive learning for Derivational Morphology}{\textbf{\large July $^{\prime}$16-April $^{\prime}$17}}{Research Project under Asst. Prof. Pawan Goyal Ghosh}{CSE Dept. IIT Kharagpur}
\item \textit Developed a graph based semi supervised approach for analysis of derivative nouns in Sanskrit.
\item Constructed character n-grams and semantic word vectors for Sanskrit dictionary. 
\item Proposed model is effective even when labelled dataset is only 5\% to that of the entire dataset.
\end{rSubsection}
%------------------------------------------------

\begin{rSubsection}{ \large Understanding code-mixing patterns in celebrity tweets}{\textbf{\large Aug $^{\prime}$17 - Dec $^{\prime}$17}}{Academic Project under Prof. Niloy Ganguly}{CSE Dept. IIT Kharagpur}

\item \textit{Abstract} -Distinguish code-mixing and code-borrowing instances in the context of celebrity tweets. We try to understand if a word from a foreign language is used by a celebrity, how likely is it to be borrowed by their followers. 
\item Analysed Sense deviation when words from foriegn languages used in Social Media context. Developed word embeddings for social media dataset.
\item Transformed word embeddings into another vector space using Hamilton's Diachronic method, compared similarity of senses and transformed vectors.
\end{rSubsection}
%------------------------------------------------

\begin{rSubsection}{ \large Graphwise - Large Scale Distributed Graph Processing}{\textbf{\large April $^{\prime}$17}}{Academic Project under Prof. Pabitra Mitra}{CSE Dept. IIT Kharagpur}

\item Used Louvain algorithm to detect communities in large networks with scala and t-SNE, a dimensionality reduction technique for data visualisation in python.

\item Reached a maximum modularity of 98.6\%. Open sourced the project on \href{https://github.com/kaustubhhiware/Graphwise}{Github}.
\end{rSubsection}
%------------------------------------------------


\end{rSection}

%------------------------------------------------

\begin{rSection}{\large Open-Source Projects}{}{More projects can be found at \textbf{\large \url{https://github.com/kaustubhhiware}}}

{\bf \large Foodspark} \hfill \textbf{Feb $^{\prime}$17} \\ 
Academic Project under Prof. Pabitra Mitra. A Django(python) webapp for online food ordering system. Hosted website on \href{http://foodkgp.pythonanywhere.com}{pythonanywhere}.
%----------------------------------------------

{\bf \large c0derunR} \hfill \textbf{Jan $^{\prime}$17} \\ 
Developed during Microsoft's Code.fun.do. A Django(python) webapp for Integrated Development Environment using HackerEarth API. Offers a static website generator reflecting on-the-go changes. \url{http://c0derunr.herokuapp.com}.
%----------------------------------------------

{\bf \large Notifyre} \hfill \textbf{Dec $^{\prime}$16} \\ 
A terminal task notifier with support of basic customzations. Supports bash, zsh and fish shell. \href{https://github.com/kaustubhhiware/ NotiFyre}{Github repository} received 28 stars so far.
%----------------------------------------------

{\bf \large Tiny-C compiler} \hfill \textbf{Nov $^{\prime}$16} \\ 
A restricted grammar C compiler made using flex,yacc and gnu assembler. A Compilers course project, - \href{https://github.com/kaustubhhiware/cOMPILER}{GitHub}.
%----------------------------------------------

{\bf \large Wordabulary} \hfill \textbf{May $^{\prime}$16} \\ 
A python pip module for grammar statistics. Support trivial document analysis. Developed as a module to validate Zipf’s law -  \href{https://github.com/kaustubhhiware/Wordabulary}{GitHub}.
%----------------------------------------------

{\bf \large Data extraction from a 2-D plot} \hfill \textbf{Feb $^{\prime}$16 - Mar $^{\prime}$16} \\ 
Developed during Inter-hall OpenSoft competition. Created a module to extract graph information from a plot and create the corresponding data table using OpenCV libraries and Tesseract -\href{https://github.com/Azad-Hall/open-soft-2015-2016}{GitHub}.
%----------------------------------------------

{\bf \large Department Information System} \hfill \textbf{Mar $^{\prime}$16} \\ 
Created a software which allows the administrator to perform managerial tasks like resource management, course handling and student graduing. Source code on \href{https://github.com/kaustubhhiware/DepInfosys}{GitHub}.
%----------------------------------------------

%------------------------------------------------
\end{rSection}

%----------------------------------------------------------------------------------------
%	INTERNSHIPS SECTION
%----------------------------------------------------------------------------------------

\begin{rSection}{Internships}

\begin{rSubsection}{\large Elanic}{\textbf{\large May $^{\prime}$17- Jul $^{\prime}$17}}{Developer Intern}{Bangalore, Karnataka}
\item Built complete CRUD operations for 2 classes, served with REST API. Abstraction written extended to 50\% of all classes.
\item Indexed 30M items, and performed exhaustive data sanitization.
\item Incorporated pre-implemented autosuggest in search bar in React, redux-based frontend.
\item Automated testing suite using Jenkins.
\item Developed a npm module \href{https://www.npmjs.com/package/urlparamify}{urlparamify} to be used in developing website, which received over 3000 downloads so far.
\end{rSubsection}

%------------------------------------------------

\end{rSection}

%----------------------------------------------------------------------------------------
%	POSITIONS OF RESPONSIBILITES SECTION
%----------------------------------------------------------------------------------------

\begin{rSection}{Positions of Responsibility}

\begin{rSubsection}{\large MetaKGP}{\textbf{\large Oct $^{\prime}$17 - Present}}{Maintainer}{IIT Kharagpur, WB}
\item Metakgp is a loose association of engineers, hackers, artists, and students from IIT Kharagpur, who collaborate on various technical and non-technical projects.
\item Managing development of multiple open source projects, with individual userbases of over 5000 students.
\end{rSubsection}
%------------------------------------------------

\begin{rSubsection}{\large coala: Language Independent Code Analysis}{\textbf{\large Mar $^{\prime}$17 - Present}}{Open source Developer}{}
\item coala provides a unified command-line interface for linting and fixing code, independent of programming languages used.
\item Sucessfully merged 4 out of 6 submitted patches so far.
\end{rSubsection}
%------------------------------------------------

\end{rSection}

%----------------------------------------------------------------------------------------
%	RELEVANT COURSES SECTION
%----------------------------------------------------------------------------------------

\begin{rSection}{Relevant Courses}

\begin{center}
\begin{tabular}{p{3.9cm} p{3.0cm} p{6.5cm} p{4.0cm}}
  Deep Learning$^+$ & Machine Learning & Natural Language Processing$^+$ & Artificial Intelligence\\
  Social Computing$^+$ & Data Analytics$^+$ & Database Management$^+$ & Operating Systems$^+$\\
  Computer Networks$^+$ & Compilers$^+$ & Computer Organisation$^+$ & Matrix Algebra\\
  Software Engineering$^+$ & Algorithms$^+$ & Formal Language \& Automata Theory & Discrete Structures\\
\end{tabular}
\footnotesize \textit{(* denotes ongoing courses, $^+$ denotes practicum included)}
\end{center}
\end{rSection}

%----------------------------------------------------------------------------------------
%	TECHNICAL SKILLS SECTION
%----------------------------------------------------------------------------------------

\begin{rSection}{Technical Skills}

\begin{tabular}{ @{} >{\bfseries}l @{\hspace{6ex}} l }
Programming Languages & Python $\bullet$ C $\bullet$ C++ $\bullet$ JavaScript $\bullet$ Java $\bullet$ Assembly $\bullet$ Scala $\bullet$ R \\
Web & HTML $\bullet$ CSS / SCSS $\bullet$ Django $\bullet$ Jekyll $\bullet$ Bootstrap $\bullet$ React \\
Tools \& Frameworks & Git $\bullet$ NodeJS $\bullet$ Shell/Fish $\bullet$ Elasticsearch $\bullet$ TravisCI $\bullet$ Heroku $\bullet$ \LaTeX \\
Databases & MongoDB $\bullet$ MySQL $\bullet$ SQLite3 \\
Operating Systems & Ubuntu/Linux $\bullet$ Windows
\end{tabular}

\end{rSection}

%----------------------------------------------------------------------------------------
%	ACHIEVEMENTS SECTION
%----------------------------------------------------------------------------------------

\begin{rSection}{Co-Curricular Achievements}
%$\bullet$ Receiving scholarship under the \textit{\textbf{Prime-Minister Scholarship Scheme (PMSS\parnote{\textbf{\textit{PMSS}} : Awarded to wards of Ex-Servicemen studying in IITs/NITs})}} since 2014 \\
%$\bullet$ \textit{\textbf{State Rank - 2$^{nd}$, \textit{School Rank - 1$^{st}$}}} in \textit{\textbf{6$^{th}$ SOF\parnote{\textbf{\textit{SOF}} : Science Olympiad Foundation} International Mathematics Olympiad}}
$\bullet$ Proposal \href{https://docs.google.com/presentation/d/1ndJbIQCxFEf__I55yOOME-a_UGWDv1KMlukVE2lbDeA/}{ResSearch} selected for final round of \href{https://drive.google.com/open?id=0B5iU6cWw36rOamZLWHZENWdlY0k}{Smart India Hackathon} amongst 8000 entries. \\
$\bullet$ Qualified for Innovation in Science Pursuit for Inspired Research {\bf \large (INSPIRE)} scholarship, awarded to only \textbf{0.01\%} students every year. \\
$\bullet$ Received scholarship under Maharashtra Talent Search Examination, obtaining a percentile of 96\%.
%$\bullet$ School-level representation at multiple quiz competitions, debate and extempores.
%$\bullet$ Have consistently been a member of school student board, holding the positions of Head Boy and Deputy Head Boy consecutively.
\end{rSection}

%----------------------------------------------------------------------------------------

%----------------------------------------------------------------------------------------
%	VOLUNTEER EXPERIENCES SECTION
%----------------------------------------------------------------------------------------

\begin{rSection}{Volunteer Experiences}

\begin{rSubsection}{\large Student Welfare Group (SWG)}{\textbf{\large Jul $^{\prime}$16 - Present}}{Mentor}{IIT Kharagpur}
\item Providing guidance to a group of 6 Undergraduate Students under SWG's \textit{Student-Mentor Program} to smoothen their academic transition as well as bolster growth in non-academic dimensions
\item Help them prepare a better study plan for upcoming semester by providing an incisive analysis of the upcoming subjects along with insightful details about diverse career opportunities to peruse
\end{rSubsection}
%---------------------------------------------------------------------

\begin{rSubsection}{\large 3 Bengal Tech Air Squadron}{\textbf{\large Jul $^{\prime}$14 - Jul $^{\prime}$16}}{Cadet}{IIT Kharagpur, WB}
\item Actively participated in numerous \textit{Tree Plantation \& Blood Donation Camps}
\item Attended the Annual NCC Camp \& organised as well as presented at the closing ceremony.
\end{rSubsection}
%---------------------------------------------------------------------


\begin{rSubsection}{\large Gopali Youth Welfare Society}{\textbf{\large Aug $^{\prime}$14 - Apr $^{\prime}$15}}{Junior Executive}{IIT Kharagpur, WB}
\item Actively participated in numerous \textit{Tree Plantation \& Blood Donation Camps}
\item Attended the Annual NCC Camp \& organised as well as presented at the closing ceremony
\end{rSubsection}
%---------------------------------------------------------------------

\end{rSection}

%----------------------------------------------------------------------------------------
%----------------------------------------------------------------------------------------
\end{document}
